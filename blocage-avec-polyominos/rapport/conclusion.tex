\chapter*{Conclusion}
\addcontentsline{toc}{chapter}{Conclusion}

En conclusion, ce projet du module d'algorithmique et programmation orientée objet (APOO) a été une expérience enrichissante et stimulante. J'ai pu mettre en pratique mes connaissances en développement orienté objet et acquérir de nouvelles compétences tout au long de sa réalisation.

L'objectif de ce projet était de concevoir et d'implémenter un jeu de blocage sur grille avec des pièces de type polyominos. J'ai réussi à créer les différentes classes nécessaires pour représenter les éléments du jeu, tels que le plateau, les joueurs, les polyominos, et à mettre en place les fonctionnalités essentielles, comme le placement des pièces, la vérification de la fin de partie et la stratégie de jeu de la machine.

L'organisation du projet a été adaptée à mes contraintes personnelles, ce qui m'a permis de travailler sur le projet en fonction de mes disponibilités. J'ai privilégié des tâches qui demandaient peu de ressources lors de séances de travail courtes et j'ai consacré des séances plus longues à des tâches plus complexes. La communication était simplifiée puisque j'ai travaillé seul sur le projet.

J'ai utilisé des ressources logicielles telles que GitHub (https://github.com/UnExtraterretres) pour la gestion du code source, Scite pour l'écriture du code en Java, et TexMaker pour la rédaction du rapport en utilisant LaTeX. Ces outils ont facilité la gestion du projet, la collaboration avec moi-même et la production d'un rapport de qualité.

En conclusion, ce projet m'a permis de consolider mes compétences en programmation orientée objet, de mettre en pratique des concepts clés tels que l'encapsulation, l'héritage et la modularité, et de développer ma logique algorithmique. J'ai également apprécié la flexibilité offerte par le projet pour m'adapter à mes contraintes personnelles.

Ce projet constitue une étape importante dans mon apprentissage de l'algorithmique et de la programmation orientée objet, et il me motive à continuer à explorer de nouveaux défis et à améliorer mes compétences en développement logiciel, tel que le LateX auquel je n'avais jamais touché.
