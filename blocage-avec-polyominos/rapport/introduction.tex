\chapter*{Introduction}
\addcontentsline{toc}{chapter}{Introduction}

Ce rapport présente le travail réalisé dans notre projet du module d'algorithmique et programmation orientée objet (APOO).

Dans le cadre du deuxième semestre de licence en sciences fondamentales de l'université de Bourgogne Franche Comté, nous sommes amenés à acquérir plusieurs compétences de développement orienté objet, telles que les classes, les objets, les constructeurs, les références, l'encapsulation, la statique ou l'instance, ainsi que la notion d'héritage.

Ce projet nous donne l'occasion de manipuler ces compétences de développement.

Le thème du sujet consiste à l'implémentation d'un jeu de blocage sur grille, avec des pièces de type polyominos. Sa réalisation doit être effectuée d'ici le 19 mai 2023, sous forme d'un ensemble de fichiers en Java (un langage de programmation orienté objet), accompagné d'un rapport, que vous êtes en train de lire.

Il nous est donc donné environ deux semaines pour réaliser, en binôme, une version fonctionnelle (même incomplète) du jeu, ainsi qu'un rapport décrivant le jeu, ses contraintes, les solutions et les choix techniques envisagés dans sa réalisation.

Afin de rendre compte de notre travail, nous allons vous décrire les axes les plus importants.
