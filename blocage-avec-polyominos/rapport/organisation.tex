\chapter*{Organisation}
\addcontentsline{toc}{chapter}{Organisation}

Ce chapitre présente l'organisation mise en place pour mener à bien le projet du module d'algorithmique et programmation orientée objet (APOO).

\section{Ressources humaines}

\subsection{Fréquence de travail}

J'ai travaillé seul sur l'ensemble du projet. La fréquence de travail était variable en raison de contraintes personnelles, administratives et logistiques. Cependant, ces interruptions dans le travail étaient prévues et anticipées, ce qui m'a permis de trouver des solutions adaptées. J'ai choisi de travailler sur mon temps libre, dès que possible, pour avancer sur le projet.

Lorsque j'avais des séances de travail de courte durée, j'ai privilégié des tâches qui demandaient peu de ressources, comme tester les fonctions pour m'assurer qu'elles remplissaient correctement leur tâche. En revanche, lors de séances plus longues, j'ai consacré du temps à des tâches qui demandaient plus de ressources, telles que réfléchir aux différentes possibilités pour certaines fonctionnalités et prendre des décisions concernant leur implémentation.

\subsection{Communication pour le travail}

Étant seul sur le projet, la communication était très simple, car tout se déroulait dans ma tête. Je n'avais pas besoin d'échanger avec d'autres personnes pour prendre des décisions ou coordonner des actions. Cela m'a permis d'avancer de manière autonome et de suivre ma propre méthodologie de travail.

\newpage
\section{Ressources logicielles}

\subsection{Pour le projet}

J'ai utilisé GitHub pour créer un dépôt distant afin de mieux suivre et organiser le projet. Cela m'a permis de sauvegarder mon travail de manière sécurisée, de suivre les modifications apportées aux fichiers et de collaborer avec moi-même en utilisant différentes branches pour développer des fonctionnalités spécifiques.

\subsection{Écriture du code}

J'ai utilisé l'éditeur Scite pour écrire le code du projet. J'ai travaillé avec Java 1.8, comme demandé dans les spécifications du projet.

\subsection{Écriture du rapport}

Pour l'écriture du rapport, j'ai utilisé TexMaker, un éditeur LaTeX convivial. LaTeX m'a permis de produire un rapport de qualité professionnelle avec une mise en page soignée, une gestion des références, des tableaux et des images, ainsi qu'une génération automatique de la table des matières et des numéros de page. Cela m'a également facilité la rédaction des sections structurées du rapport.
