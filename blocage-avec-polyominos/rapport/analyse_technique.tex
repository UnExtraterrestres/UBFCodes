\chapter{Analyse technique du sujet proposé}

\section{Présentation rapide}

Le jeu oppose deux personnes, ici humain contre machine, sur un plateau de forme rectangulaire. Les joueurs ont des pièces, des polyominos qu'ils placent chacun leur tour. La fin du jeu est décidée par un blocage, c'est-à-dire lorsque l'un des joueurs ne peut plus poser aucune de ses pièces, il est alors perdant. Les affichages se font sur la console. Le mode de jeu demandé est humain contre machine. La stratégie de jeu de la machine est libre. Au niveau du code, il nous est imposé de travailler en programmation orientée objet et d'utiliser l'encapsulation.

\newpage
\section{Définitions}

\textbf{Programmation orientée objet :} Un paradigme de programmation qui organise les programmes autour d'objets, qui représentent des entités ayant des caractéristiques (attributs) et des comportements (méthodes). L'approche orientée objet favorise la modularité, la réutilisabilité et la flexibilité du code. Par exemple on va définir une chaise comme ayant un nombre de pieds des matériaux un coût une résistance, et d'autre part une personne avec son nom sa taille son poids, puis on défini leurs comportements : la personne pourra s'assoir sur la chaise, la déplacer mais pas la chaise sur la personne évidemment.

\textbf{Encapsulation :} Un principe de la programmation orientée objet qui consiste à regrouper les données (attributs) et les méthodes qui les manipulent dans une même entité appelée objet. L'encapsulation permet de protéger les données internes d'un objet en les rendant accessibles uniquement par des méthodes spécifiques. En reprenant l'exemple de la classe Personne décrite ci-dessus, on peut imaginer que celle-ci possède un ensemble de meubles, dont des chaises, on dit alors qu'il y encapsulation.

\textbf{Polyominos :} Un assemblage de carrés, collés les uns aux autres par leurs côtés, mais ne pouvant pas être assemblés par un coin. Les polyominos existent dans différentes tailles, les plus courants étant les tétrominos (pièces utilisées dans Tetris) et les dominos. Dans ce jeu, nous utiliserons des polyominos composés de 2, 3 ou 4 carrés.
